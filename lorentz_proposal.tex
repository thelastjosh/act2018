\documentclass{article}
\usepackage{amsmath, amsthm, amssymb, comment}
\usepackage{color}
\usepackage[all]{xy}
\usepackage{enumitem}
\title{Workshop proposal: Applied Category Theory}
\date{\today}

% editing definitions
\newcommand{\grayout}[1]{{\color{gray}#1}}
\newcommand{\redout}[1]{{\color{red}#1}}
\newcommand{\marginnote}[1]{\marginpar{\footnotesize \color{blue}#1}}

\begin{document}
\maketitle

\section{Organizers}
\begin{enumerate}
\item John Baez (math)
\item Bob Coecke (physics and linguistics) 
\item Brendan Fong (dynamical systems)
\item Aleks Kissinger, main contact (computer science)
\item Joshua Tan, main contact (computer science)
\end{enumerate}

\section{Scientific case}
\subsection{Scientific background}
Category theory was developed in the 1940s to translate ideas from one field of mathematics, e.g. topology, to another field of mathematics, e.g. algebra. More recently, category theory has become an unexpectedly useful and economical tool for modeling a range of different disciplines, including programming language theory \cite{abramsky}, quantum mechanics \cite{coecke}, systems biology \cite{cardelli}, complex networks \cite{baez}, database theory \cite{rosebrugh}, and dynamical systems \cite{spivak}.

A category consists of a collection of objects together with a collection of maps between those objects, satisfying certain rules. Topologists and geometers use category theory to describe the passage from one mathematical structure to another, while pure category theorists study categories as their own mathematical structures. In computer science and physics, many categorical constructions (e.g. topoi, or Stone duality, or compact closed categories) are used to give a formal semantics of domain-specific phenomena (e.g. recursive types \cite{simpson}, or automata \cite{pippenger}, or quantum protocols \cite{abramsky_coecke}).\footnote{The categorical semantics is often preferable to set- or type-theoretic semantics in some way: for example, compact closed categories have an elegant graphical language in terms of string diagrams.} In the applied category theory community, a long-articulated vision understands categories as mathematical workspaces for all sciences, similar to how they are used in topology and geometry \cite{spivak_book, earlier_citation?}. While it is too early to say whether this vision will bear fruit, we do believe that category theory has the potential to bridge \emph{specific} different fields, and moreover that developments in such fields (e.g. automata) can be transferred successfully into other fields (e.g. systems biology) \emph{through} category theory.

Most extant applications of category theory outside of pure mathematics have used categories to model some phenomena, in the tradition of formal semantics. The most recent workshop on applied category theory was held at Dagstuhl, Germany, in 2015, where ``the over-arching research theme was to develop categorical methods as a unified approach to the modeling of complex systems, and category theory as a paradigm for mathematical modeling and applied science'' \cite{dagstuhl}. But one of the observations made at Dagstuhl, was that ``a weakness of semantics/CT is that the definitions play a key role. Having the right definitions makes the theorems trivial, which is the opposite of hard subjects where they have combinatorial proofs of theorems (and simple definitions). [...] In general, the audience agrees that people see category theorists only as reconstructing the things they knew already, and that is a disadvantage, because we do not give them a good reason to care enough'' \cite{selinger_at_dagstuhl}.

In this workshop, we wish to articulate a natural response: instead of treating the reconstruction as a weakness, we should treat categorification as a natural consequence of transferring and integrating knowledge across disciplines, in the sense of a mathematical workspace. The greatest barrier to knowledge transfer between disciplines is the one of language: jargon in one discipline renders it incomprehensible to those outside the research community. The restructuring employed in applied category theory elucidates common themes across disciplines, and rephrases it in an attempt at a common language. Indeed, the drive for a common language and comparison of similar structures in algebra and topology is what led to the development category theory in the first place, and recent hints show that this approach is not only useful between mathematical disciplines, but between scientific ones as well. For example, the `Rosetta Stone' of Baez and Stay demonstrates how cartesian closed categories capture the common structure between logic, computation, and physics \cite{baez_rosetta}. 

%Through the curry  The efficacy of this has been demonstrated \redout{We believe... this is the future? ... this approach has been underrepresented? ... this topic has been largely unexplored? ... the difference is largely psychological, like the difference between classical and modern geometry? ... cited in early examples, like Baez's `rosetta stone' \cite{baez_rosetta}? ... is the language just convenient for manipulation, or are there real \emph{theorems} that go between fields?}

% The first and most successful example of this was the use of category theory to transmit ideas from mathematics into computer science, for example through monads in functional programming, or through the Curry-Howard-Lambek correspondence between intuitionistic logic, typed lambda calculus, and cartesian closed categories.

\subsection{Specific challenges and outcomes}
This workshop will bring together both theorists and practitioners from a wide variety of disciplines to work on new applications of category theory in (1) dynamical systems and networks, (2) systems biology, and (3) cognition and AI, with a special focus on developing a community of early-stage researchers in applied category theory, and on fostering focused dialogue between researchers working on different applications. It will consist of a 5-day workshop week, an attached 3-day tutorial weekend immediately before, and a 3-month online seminar for PhD students called the ``Kan Extension Lab''.

Some of the specific challenges and outcomes we wish to address include:
\begin{enumerate}
  \item Computability: while category theory provides a firm foundation for reasoning as it occurs across many disciplines, to be applied rather than merely applicable requires tools that permit applied practitioners to take advantage of this structure. One avenue is accessible software packages that implement category theoretic reasoning.  Vicary et al's popular online proof assistant Globular, based on higher category theory, demonstrates the demand for and utility of such software packages; it is crucial to the outreach of applied category theory that work continues in this vein.
\item Communication: applied category theory depends on finding open problems in other fields where CT can make a contribution, but how should category theorists communicate with practitioners in these other fields? Moreover, how can we begin communicating with industrial partners, in order to develop industrial applications? % Are there general criteria to pick out problems and contexts in which category theory can be useful?
\item Pedagogy: one of the open problems discussed at Dagstuhl was the perceived and actual difficulty of category theory. Despite the flexibility and expressiveness of categorical tools in mathematics and computer science, the perceived difficulty of category theory has hindered wider acceptance of the formalism in other areas of interest. Different approaches were suggested, including focusing on automated theorem proving. We plan on addressing this problem over the tutorial weekend, and through the organization of the ``Kan Extension Lab''.
\end{enumerate}

Other challenges, specific to each application area, include \redout{???} (dynamical systems), \redout{???} (dynamical systems), \redout{???} (systems biology), and \redout{???} (cognition and AI).

Our workshop will be considered a success if it results in joint research between researchers specializing in different applications (e.g. physics and biology, or economics and AI) or in research that carries over techniques from one application domain of category theory to another, and if the workshop introduces new researchers into the field.

\subsection{Connection to the Dutch research community}
In the Dutch applied category theory community, a number of people use category theory  in the context of coalgebras, especially for software verification. Bart Jacobs has worked on this, along with Helle (Delft), Juriaan Rot (Radboud), and Jan Rutten (Amsterdam). Michael Moortgat (Utrecht) and \redout{Martha Lewis (???)} work on linguistic applications. Aleks Kissinger (Radboud) works on quantum algorithms and graph rewriting systems using category theory. In pure category theory and topos theory, Ieke Moerdijk (Utrecht) developed many of the foundations. Klaas Landsmen (Radboud) works on topos theory, operator algebras, and quantum theory. In categorical logic are Sonja Smets and Alexandru Baltag (Amsterdam); both are friendly to applications of category theory, e.g. categorical quantum mechanics.

\section{Program}
\subsection{Workshop week}
The workshop highlights three particular applications of category theory: (1) to dynamical systems and networks, (2) to systems biology, and (3) to cognition and AI. While there will be a short introductory lecture for each application domain, the afternoons will intermix all three applications by focusing on common techniques (Monday and Wednesday), computational tools (Tuesday), and common problems and goals (Thursday afternoon) across all three. On Thursday morning, there will be a half-day Highlights session of 8-minute talks.

\begin{table}[h!]
\begin{center}
%\tiny
\begin{tabular}{|c||p{.15\linewidth}|p{.15\linewidth}|p{.15\linewidth}|p{.15\linewidth}|p{.15\linewidth}|}
\hline
Time & Monday & Tuesday & Wednesday & Thursday & Friday \\ \hline
9:00 - 9:30 & Arrival & & & & \\ \cline{2-4} \cline{6-6}
9:30 - 10:00 & Welcome & Sys. Bio. & Cogn. + AI &  & talk? \\ \cline{2-2}
10:00 - 10:30 & Dyn. Sys. & (Krivine)  & (Coecke) & Highlights & \\ \cline{3-4} \cline{6-6}
10:30 - 11:00 & (Baez) & \emph{break} & \emph{break} & forum & \emph{break} \\  \cline{2-4} \cline{6-6}
11:00 - 11:30 & 2 talks & 2 talks & 2 talks & & 2 talks \\ \cline{2-4} \cline{6-6}
11:30 - 12:30 & Discussion & Discussion & Discussion & & Discussion \\ 
&&&&& \\\hline
12:30 - 2:00 & Lunch & Lunch & Lunch & Lunch & Lunch \\
&&&&& \\
&&&&& \\ \hline
2:00 - 3:00 & Sheaves & Computing & Monoidal & & talk? \\
& (Abramsky) & (Vicary) & (?) & Problem & \\ \cline{2-4} \cline{6-6}
3:00 - 3:45 & \emph{break} & \emph{break} & Discussion & Session & Discussion \\ \cline{2-4} \cline{6-6}
3:45 - 4:45 & Discussion & Discussion & Boat & & Closing \\ 
4:45 - 5:30 & & & trip and & & \\ \cline{2-3} \cline{5-6}
& Wine and cheese & & dinner & &  \\ \hline
\end{tabular}
\end{center}
% \caption{Program for week 1.}
\end{table}%

\begin{table}[h!]
\begin{center}
\begin{tabular}{|c|c|c|} \hline
scheduled time for & per day & week total \\ \hline
lectures & 2.5 hrs & 12.5 hrs  \\
discussion & 2.3 hrs & 11.75 hrs \\
lunch/break & 2.1 hrs & 10.5 hrs \\ \hline
\end{tabular}
\end{center}
\end{table}%

Each working day will include one keynote lecture during the morning that sets the stage for the day, followed by two to three 15-minute talks that delve into specific aspects of content in the keynote lecture. For example, on Wednesday, there will be a survey lecture by \redout{Bob Coecke} on computational category theory, followed by two 15-minute talks on \redout{???} and \redout{???}. These will be followed by an additional talk in the early afternoon, which will present active areas of research in these topics. Each morning and afternoon will be closed by a discussion session or an extended coffee break.

% during these sessions the attendants will split into several groups according to the main thematic areas that had been identified on the first day. Suggested application areas include (quantum) computation, physics, biology, complex systems, economic, social and cognitive science, and linguistics. Indeed, some of the items span more than one discipline, e. g. game theory, and the list is definitively not exhaustive.

\subsection{Tutorial weekend}
Immediately prior to the workshop, we will organize a 3-day weekend of tutorials targeted at graduate students and postdocs, though we envision more senior researchers who wish to broaden their horizons will also be interested in attending.

Two keynote lectures at the tutorials will be given by \redout{John Baez and Jamie Vicary}.

To supplement the tutorial weekend, we will host an online ``Kan Extension Lab'' for graduate students prior to the workshop, whose participants will present the results of their work in the lab either at the tutorial weekend or at the workshop itself. The output of the Extension Lab will be published in a special issue of \redout{???}.

\section{Participants}
The estimated number of participants is 55. At the time of writing of this proposal, the planned workshop already has ?? confirmed participants with affiliations in the Netherlands, the UK, the US, France, Germany, Italy, and [...].

Confirmed participants for tutorial weekend: \redout{??}. 

Junior/senior ratio: \redout{??}.

\begin{enumerate}
\item Samson Abramsky (Oxford), prof, computer science
\item John Baez (UC Riverside), prof, math
\item Bob Coecke (Oxford), prof, computer science
\item Brendan Fong (MIT), postdoc, computer science
\item Joshua Tan (Oxford), PhD, computer science
\item \redout{... [copy over from spreadsheet when complete.]}
\end{enumerate}

\section{Factsheet}
A separate factsheet has been attached to this application.

\section{Budget}
A separate budget has been attached to this application.

\bibliographystyle{plain}
\bibliography{lorentz} 

\end{document}
