\documentclass{article}
\usepackage{amsmath, amsthm, amssymb, comment}
\usepackage{color}
\usepackage[all]{xy}
\usepackage{enumitem}
\title{Workshop proposal: Applied Category Theory}
\date{\today}

% editing definitions
\newcommand{\grayout}[1]{{\color{gray}#1}}
\newcommand{\redout}[1]{{\color{red}#1}}
\newcommand{\marginnote}[1]{\marginpar{\footnotesize \color{blue}#1}}

\begin{document}
\maketitle

\section{Organizers}
\begin{enumerate}
\item John Baez (math)
\item Bob Coecke (physics and linguistics) 
\item Brendan Fong (dynamical systems)
\item Aleks Kissinger, main contact (computer science)
\item Joshua Tan, main contact (computer science)
\end{enumerate}

\section{Scientific case}
\subsection{Scientific background}
Category theory was developed in the 1940s to translate ideas from one field of mathematics, e.g. topology, to another field of mathematics, e.g. algebra. More recently, category theory has become an unexpectedly useful and economical tool for modeling a range of different disciplines, including programming language theory \cite{abramsky}, quantum mechanics \cite{coecke}, systems biology \cite{cardelli}, complex networks \cite{baez}, database theory \cite{rosebrugh}, and dynamical systems \cite{spivak}.

A category consists of a collection of objects together with a collection of maps between those objects, satisfying certain rules. Topologists and geometers use categories as mathematical workspaces, while pure category theorists study them as their own mathematical structures. In computer science and physics, many categorical constructions (e.g. topoi, or Stone duality, or compact closed categories) are used to give a formal semantics of domain-specific phenomena (e.g. recursive types \cite{simpson}, or automata \cite{pippenger}, or quantum protocols \cite{abramsky_coecke}).\footnote{The categorical semantics is often preferable to set- or type-theoretic semantics in some way: for example, compact closed categories have an elegant graphical language in terms of string diagrams.} In the applied category theory community, a long-articulated vision understands categories as mathematical workspaces for all sciences, not just topology and geometry \cite{spivak_book}. While it is too early to say whether this vision will bear fruit, we do believe that category theory has the potential to bridge \emph{specific} different fields, and moreover that developments in such fields (e.g. dynamical systems) can be transferred successfully into other fields (e.g. systems biology) \emph{through} category theory.

Most extant applications of category theory outside of pure mathematics have used categories to \emph{model} some phenomena, in the tradition of formal semantics. The most recent workshop on applied category theory was held in 2015 at Dagstuhl, Germany, wherein ``the over-arching research theme was to develop categorical methods as a unified approach to the modeling of complex systems, and category theory as a paradigm for mathematical modeling and applied science'' \cite{dagstuhl}. It is another, further step to use category theory as a \emph{medium} for knowledge representation and integration, in the sense of a mathematical workspace. \redout{We believe... this is the future? ... this approach has been underrepresented? ... this topic has been largely unexplored? ... the difference is largely psychological, like the difference between classical and modern geometry? ... cited in early examples, like Baez's `rosetta stone' \cite{baez_rosetta}? ... is the language just convenient for manipulation, or are there real \emph{theorems} that go between fields?}

% The first and most successful example of this was the use of category theory to transmit ideas from mathematics into computer science, for example through monads in functional programming, or through the Curry-Howard-Lambek correspondence between intuitionistic logic, typed lambda calculus, and cartesian closed categories.

 % The goals of the workshop are (1) to create a more cohesive and connected ACT community, especially among early-stage researchers, (2) to develop existing and new applications of category theory, and (3) to try to outline common goals and open problems for the field, i.e. across multiple applications.

\subsection{Precedence}
Dagstuhl?

\subsection{Specific challenges and outcomes}
This workshop will bring together both theorists and practitioners from a wide variety of disciplines to work on new applications of category theory in (1) dynamical systems and networks, (2) systems biology, and (3) cognition and AI, with a special focus on developing a community of early-stage researchers in applied category theory, and on fostering fresh dialogue between researchers working on different applications.

Some of the specific challenges and outcomes we wish to address include:
\begin{enumerate}
\item Computability: \redout{BRENDAN}
\item Pedagogy: despite the flexibility and expressiveness of categorical tools in mathematics and computer science, the perceived difficulty of category theory has hindered wider acceptance of the formalism. As a consequence, many researchers in different communities share the feeling of under-exploitation of the potentialities of category theory to their areas of interest. We plan on addressing this problem over the tutorial weekend, and through the organization of the ``Kan Extension Lab''.
\end{enumerate}

Other challenges, specific to each application area, include \redout{???} (dynamical systems), \redout{???} (dynamical systems), \redout{???} (systems biology), and \redout{???} (cognition and AI).

Our workshop will be considered a success if it results in joint research between researchers specializing in different applications (e.g. physics and biology, or economics and AI) or in research that carries over techniques from one application domain of category theory to another, and if the workshop introduces new people into the field.

\subsection{Connection to the Dutch research community}
\redout{ALEKS: in the Dutch applied category theory community... this would be Bart + Aleks at Nijmegen, plus Martha Lewis, plus some others.}

\section{Program}
\subsection{Workshop week}
The workshop highlights three particular applications of category theory: (1) to dynamical systems and networks, (2) to systems biology, and (3) to cognition and AI. While there will be a short introductory lecture for each application domain, the afternoons will intermix all three applications by focusing on common techniques (Monday and Wednesday), computational tools (Tuesday), common problems (Thursday morning), and common goals (Thursday afternoon) across all three.

\begin{table}[h!]
\begin{center}
\begin{tabular}{|c||p{.15\linewidth}|p{.15\linewidth}|p{.15\linewidth}|p{.15\linewidth}|p{.15\linewidth}|}
\hline
Time & Monday & Tuesday & Wednesday & Thursday & Friday \\ \hline
9:00 - 9:30 & Arrival at & & & & \\ \cline{3-6}
9:30 - 10:00 & Lorentz & Systems & Cognition & ? & \\ \cline{2-2}
10:00 - 10:30 & Welcome & Biology & and AI & ? & \\ \cline{2-2}
10:30 - 11:30 & Dynamical & (Krivine) & (Coecke) & ? & ? \\  \cline{3-6}
11:30 - 12:15 & Systems (Baez) & Discussion & Discussion & Discussion & Discussion \\ \hline
12:15 - 2:00 & Lunch & Lunch & Lunch & Lunch & Lunch \\
&&&&& \\ \hline
2:00 - 3:00 & Sheaves & Comput- & Monoidal & ??? & ??? \\
& (Abramsky) & ation (Vicary) & (?) & & \\ \hline
3:00 - 3:45 & Discussion & Discussion & Discussion & Discussion & Discussion \\ \hline
3:45 - 4:45 & & & Boat & & \\ 
4:45 - 5:30 & & & trip and & & \\ \cline{2-3} \cline{5-6}
& Wine and cheese & & dinner & &  \\ \hline
\end{tabular}
\end{center}
% \caption{Program for week 1.}
\end{table}%

Each working day will include one keynote lecture during the morning that sets the stage for the day, followed by a ``lightning round'' of four 15-minute talks that delve into specific aspects of content in the keynote lecture. For example, on Tuesday, there will be a survey lecture by \redout{Jason Morton} on computational category theory, followed by talks on \redout{???}, \redout{???}, and \redout{???}. These will be followed by four additional 15-minute talks in the early afternoon, which will present active areas of research in these topics. Each morning and afternoon will be closed by a problem session or an extended coffee break. 

Overview: 

% during these sessions the attendants will split into several groups according to the main thematic areas that had been identified on the first day. Suggested application areas include (quantum) computation, physics, biology, complex systems, economic, social and cognitive science, and linguistics. Indeed, some of the items span more than one discipline, e. g. game theory, and the list is definitively not exhaustive.

\subsection{Tutorial weekend}
Immediately prior to the workshop, we will organize a 3-day weekend of tutorials targeted at graduate students and postdocs, though we envision more senior researchers who wish to broaden their horizons will also be interested in attending.

To supplement the tutorial weekend, we will host an online ``Kan Extension Lab'' for graduate students prior to the workshop, whose participants will then be invited to speak at either the tutorial weekend or at the workshop itself. The output of the Extension Lab will be published in a special issue of \redout{???}.

\section{Participants}
The estimated number of participants is 55. At the time of writing of this proposal, the planned workshop already has ?? confirmed participants with affiliations in the Netherlands, the UK, the US, and [...].

Confirmed participants for tutorial weekend: ??. 

Junior/senior ratio: ??.

\begin{enumerate}
\item First Last (Affiliation), prof/PhD/postdoc, subject of expertise
\item Samson Abramsky (Oxford), prof, computer science
\item John Baez (UC Riverside), prof, math
\item Bob Coecke (Oxford), prof, computer science
\item Brendan Fong (MIT), postdoc, computer science
\item Joshua Tan (Oxford), PhD, computer science
\item ... [copy over from spreadsheet when complete.]
\end{enumerate}

\section{Factsheet}
A separate factsheet has been attached to this application.

\section{Budget}
A separate budget has been attached to this application.

\end{document}