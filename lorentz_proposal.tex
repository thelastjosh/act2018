\documentclass{article}
\usepackage{amsmath, amsthm, amssymb, comment}
\usepackage{color}
\usepackage[all]{xy}
\usepackage{enumitem}
\title{Workshop proposal: Applied Category Theory}
\date{\today}

% editing definitions
\newcommand{\grayout}[1]{{\color{gray}#1}}
\newcommand{\redout}[1]{{\color{red}#1}}
\newcommand{\marginnote}[1]{\marginpar{\footnotesize \color{blue}#1}}

\begin{document}
\maketitle

\section{Organizers}
\begin{enumerate}
\item John Baez (math)
\item Bob Coecke (physics and linguistics) 
\item Brendan Fong (dynamical systems)
\item Aleks Kissinger (computer science)
\item Martha Lewis (linguistics)
\item Joshua Tan, main contact (computer science)
\end{enumerate}

\section{Scientific case}
\subsection{Scientific background}
Category theory was developed in the 1940s to translate ideas from one field of mathematics, e.g. topology, to another field of mathematics, e.g. algebra. More recently, category theory has become an unexpectedly useful and economical tool for modeling a range of different disciplines, including programming language theory \cite{moggi91}, quantum mechanics \cite{abramsky09}, systems biology \cite{rosen58}, complex networks \cite{baez15}, database theory \cite{rosebrugh03}, and dynamical systems \cite{spivak16}.

A category consists of a collection of objects together with a collection of maps between those objects, satisfying certain rules. Topologists and geometers use category theory to describe the passage from one mathematical structure to another, while category theorists are also interested in categories for their own sake. In computer science and physics, many types of categories (e.g. topoi or monoidal categories) are used to give a formal semantics of domain-specific phenomena (e.g. automata \cite{arbib05}, or regular languages \cite{pippenger97}, or quantum protocols \cite{abramsky09}).\footnote{The categorical semantics is often preferable to set- or type-theoretic semantics in some way: for example, compact closed categories have an elegant graphical language in terms of string diagrams.} In the applied category theory community, a long-articulated vision understands categories as mathematical workspaces for the experimental sciences, similar to how they are used in topology and geometry \cite{spivak}. This has proved true in certain fields, including computer science and mathematical physics, and we believe that these results can be extended in an exciting direction: we believe that category theory has the potential to bridge specific different fields, and moreover that developments in such fields (e.g. automata) can be transferred successfully into other fields (e.g. systems biology) \emph{through} category theory. Already, for example, the categorical modeling of quantum processes has helped solve an important open problem in natural language processing \cite{sadrzadeh13}.

In this workshop, we want to instigate a multi-disciplinary research program in which concepts, structures, and methods from one discipline can be reused in another. Tangibly and in the short-term, we will bring together people from different disciplines in order to write an expository/survey paper proposing the research program. In formulating this research program, we are motivated by recent successes where category theory was used to model a wide range of phenomena across many disciplines, e.g. open dynamical systems (including open Markov processes and open chemical reaction networks), entropy and relative entropy \cite{leinster11}, and descriptions of computer hardware \cite{ghica16}. Several talks will address some of these new developments. But we are also motivated by an open problem in applied category theory, one which was observed at the most recent workshop in applied category theory (Dagstuhl, Germany, in 2015): ``a weakness of semantics/CT is that the definitions play a key role. Having the right definitions makes the theorems trivial, which is the opposite of hard subjects where they have combinatorial proofs of theorems (and simple definitions). [...] In general, the audience agrees that people see category theorists only as reconstructing the things they knew already, and that is a disadvantage, because we do not give them a good reason to care enough'' \cite[pg. 61]{dagstuhl14}. In this workshop, we wish to articulate a natural response: instead of treating the reconstruction as a weakness, we should treat the use of categorical concepts as a natural part of transferring and integrating knowledge across disciplines. The restructuring employed in applied category theory cuts through jargon, helping to elucidate common themes across disciplines. Indeed, the drive for a common language and comparison of similar structures in algebra and topology is what led to the development category theory in the first place, and recent hints show that this approach is not only useful between mathematical disciplines, but between scientific ones as well. For example, the `Rosetta Stone' of Baez and Stay demonstrates how symmetric monoidal closed categories capture the common structure between logic, computation, and physics \cite{baez09}.

% Most extant applications of category theory outside of pure mathematics have used categories to model phenomena in the tradition of formal semantics. The most recent workshop on applied category theory was held at Dagstuhl, Germany, in 2015, where ``the over-arching research theme was to develop categorical methods as a unified approach to the modeling of complex systems, and category theory as a paradigm for mathematical modeling and applied science'' \cite{dagstuhl14}. One of the observations made at Dagstuhl as that ``a weakness of semantics/CT is that the definitions play a key role. Having the right definitions makes the theorems trivial, which is the opposite of hard subjects where they have combinatorial proofs of theorems (and simple definitions). [...] In general, the audience agrees that people see category theorists only as reconstructing the things they knew already, and that is a disadvantage, because we do not give them a good reason to care enough'' \cite[pg. 61]{dagstuhl14}.



% The first and most successful example of this was the use of category theory to transmit ideas from mathematics into computer science, for example through monads in functional programming, or through the Curry-Howard-Lambek correspondence between intuitionistic logic, typed lambda calculus, and cartesian closed categories.

\subsection{Specific challenges and outcomes}
This workshop will bring together both theorists and practitioners from a wide variety of disciplines to work on new applications of category theory in (1) dynamical systems and networks, (2) systems biology, and (3) cognition and AI, with a special focus on developing a community of early-stage researchers in applied category theory, and on fostering focused dialogue between researchers working on different applications. It will consist of a 5-day workshop week, an attached 3-day tutorial weekend immediately before, and a 3-month online seminar for PhD students called the ``Kan Extension Lab''.

Some of the specific challenges and outcomes we wish to address include:
\begin{enumerate}
  \item Tool support: while category theory provides a firm foundation for reasoning as it occurs across many disciplines, to be applied rather than merely applicable requires tools that permit applied practitioners to take advantage of this structure. One avenue is accessible software packages that implement category theoretic reasoning. Vicary et al's popular online proof assistant Globular, based on higher category theory, demonstrates the demand for and utility of such software packages; it is crucial to the outreach of applied category theory that work continues in this vein.
\item Communication: applied category theory depends on finding open problems in other fields where CT can make a contribution, but how should category theorists communicate with practitioners in these other fields? Moreover, how can the research community develop deeper partnerships with industrial partners in order to develop industrial applications, e.g. product lifecycle management tools or models of interoperability with aerospace manufacturers such as Airbus and Dassault. % Are there general criteria to pick out problems and contexts in which category theory can be useful?
\item Pedagogy: one of the open problems discussed at Dagstuhl was the perceived and actual difficulty of category theory. Despite the flexibility and expressiveness of categorical tools in mathematics and computer science, the perceived difficulty of category theory has hindered wider acceptance of the formalism in other areas of interest, e.g. to students in areas outside of math and CS. Different approaches were suggested, including focusing on automated theorem proving. We plan on addressing this problem over the tutorial weekend, and through the organization of the ``Kan Extension Lab''.
\end{enumerate}

% Other challenges, specific to each application area, include \redout{???} (dynamical systems), \redout{???} (systems biology), and \redout{???} (cognition and AI).

Our workshop will be considered a success if it results in joint research between researchers specializing in different applications (e.g. physics and biology, or economics and AI) or in research that carries over techniques from one application domain of category theory to another, and if the workshop introduces new researchers into the field.

\subsection{Connection to the Dutch research community}
In the Dutch applied category theory community, a number of people use category theory  in the context of coalgebras, especially for software verification. Bart Jacobs has worked on this, along with Helle Hvid Hansen (Delft), Juriaan Rot (Radboud), and Jan Rutten (Amsterdam). Ralf Hinze (Nijmegen) applies category-theoretic methods to functional programming. Michael Moortgat (Utrecht) and Martha Lewis (Radboud) work on linguistic applications. Aleks Kissinger (Radboud) works on quantum algorithms and graph rewriting systems using category theory. In pure category theory and topos theory, Ieke Moerdijk (Utrecht) developed many of the foundations. Klaas Landsmen (Radboud) works on topos theory, operator algebras, and quantum theory. Dutch researchers in categorical logic include Sonja Smets and Alexandru Baltag (Amsterdam); both are friendly to applications of category theory, e.g. categorical quantum mechanics.

\section{Program}
\subsection{Workshop week}
The workshop highlights three particular applications of category theory: (1) to dynamical systems and networks, (2) to systems biology, and (3) to cognition and AI. While there will be a short introductory lecture for each application domain, the afternoons will intermix all three applications by focusing on common techniques (Monday and Tuesday), computational tools (Wednesday), and common problems and goals (Thursday afternoon) across all three. On Thursday morning, there will be a half-day Highlights session of 8-minute talks.

\begin{table}[h!]
\begin{center}
%\tiny
\begin{tabular}{|c||p{.15\linewidth}|p{.15\linewidth}|p{.15\linewidth}|p{.15\linewidth}|p{.15\linewidth}|}
\hline
Time & Monday & Tuesday & Wednesday & Thursday & Friday \\ \hline
9:00 - 9:30 & Arrival & & & & \\ \cline{2-4} \cline{6-6}
9:30 - 10:00 & Welcome & Cogn. + AI & Sys. Bio. &  & Industrial \\ \cline{2-2}
10:00 - 10:30 & Dyn. Sys. & (Jacobs)  & (Krivine) & Highlights & (TBD) \\ \cline{3-4} \cline{6-6}
10:30 - 11:00 & (Baez) & \emph{break} & \emph{break} & forum & \emph{break} \\  \cline{2-4} \cline{6-6}
11:00 - 11:30 & 2 talks & 2 talks & 2 talks & & 2 talks \\ \cline{2-4} \cline{6-6}
11:30 - 12:30 & Discussion & Discussion & Discussion & & Discussion \\ 
&&&&& \\\hline
12:30 - 2:00 & Lunch & Lunch & Lunch & Lunch & Lunch \\
&&&&& \\
&&&&& \\ \hline
2:00 - 3:00 & Entropy & KR & Computing & & New Direct. \\
& (Leinster) & (Abramsky) & (Moss) & Problem & (TBD) \\ \cline{2-4} \cline{6-6}
3:00 - 3:45 & \emph{break} & \emph{break} & Discussion & Session & Discussion \\ \cline{2-4} \cline{6-6}
3:45 - 4:45 & Discussion & Discussion & Boat & & Closing \\ 
4:45 - 5:30 & & & trip and & & \\ \cline{2-3} \cline{5-6}
& Wine and cheese & & dinner & &  \\ \hline
\end{tabular}
\end{center}
% \caption{Program for week 1.}
\end{table}%

\begin{table}[h!]
\begin{center}
\begin{tabular}{|c|c|c|} \hline
scheduled time for & per day & week total \\ \hline
lectures & 2.5 hrs & 12.5 hrs  \\
discussion & 2.3 hrs & 11.75 hrs \\
lunch/break & 2.1 hrs & 10.5 hrs \\ \hline
\end{tabular}
\end{center}
\end{table}%

Each working day will include one keynote lecture during the morning that sets the stage for the day, followed by two to three 15-minute talks that delve into specific aspects of content in the keynote lecture. For example, on Tuesday, there will be a survey lecture by \redout{Bart Jacobs} on category theory for AI, followed by two 15-minute talks delving into Bayesian probability and linguistics. These will be followed by an additional talk in the early afternoon, which will present active areas of research in these topics. Each morning and afternoon will be closed by a discussion session or an extended coffee break. During each discussion session, the participants will be encouraged to break into smaller groups, each led by one of the presenters or a senior researcher, that will focus on developing some of the problems and issues raised by the presenters in their lectures.

% during these sessions the attendants will split into several groups according to the main thematic areas that had been identified on the first day. Suggested application areas include (quantum) computation, physics, biology, complex systems, economic, social and cognitive science, and linguistics. Indeed, some of the items span more than one discipline, e. g. game theory, and the list is definitively not exhaustive.

\subsection{Tutorial weekend}
Immediately prior to the workshop, we will organize a 3-day weekend of tutorials targeted at graduate students and postdocs, though we envision that researchers in other fields who wish to learn more about applied category theory will also be interested in attending.

\begin{table}[h!]
\begin{center}
%\tiny
\begin{tabular}{|c||p{.15\linewidth}|p{.15\linewidth}|p{.15\linewidth}|}
\hline
Time & Friday & Saturday & Sunday \\ \hline
9:00 - 10:00 & Arrival & & \\
10:00 - 11:00 & Tutorial: Diagrams & Keynote: Baez  & Keynote: Pawel \\ 
11:00 - 12:00 & Case Study: Globular & & \\ 
12:00 - 12:30 & Case Study: FQL & Case Study: CPS & Discussion \\ 
&&& \\\hline
12:30 - 2:00 & Lunch & Lunch & Lunch \\
&&& \\
&&& \\ \hline
2:00 - 3:00 & Tutorial: PROPs & Tutorial: Globular & Tutorial: TBD \\
3:00 - 4:00 & \emph{break} & \emph{break} & \emph{break} \\
4:00 - 5:00 & Case Study: NNets & Case Study: Physics & Discussion  \\ 
5:00 - evening & & Party & Wine \& Cheese \\ \hline
\end{tabular}
\end{center}
% \caption{Program for week 1.}
\end{table}%

Two keynote lectures at the tutorials will be given by \redout{John Baez and Pawel Sobocinski}. Other events include introductory tutorials on diagrammatic reasoning, an introduction to Globular, a ``user's guide to operads and PROPs'', case studies of how to use these tools in applications, a discussion of how to find new applications, and a working session dedicated to a constructing a survey of applied category theory (see below).

%If they�re really industry people, we�d need an introductory tutorial on diagrammatic reasoning. Depending on how applied they are, we may need a categories for applications tutorial as well. For PhD students, things like Pawel�s linear algebra stuff would be good. Also on Pawel�s stuff, I think for PhD / junior researchers, a �users guide to PROPs� would be very useful, Pawel and coauthors are doing really useful stuff and their skills with PROPs could be used by others. Something on using the network theory stuff would also be great.
%Also, it would be useful to have somebody applications oriented try and explain to us how they think about stuff. For example somebody with knowledge of these various diagram languages used in applications could explain how they�re used in practice, what sort of problems they�re really interested in, what they can / can�t do etc. We�d have to think quite hard about how to find the right person / people here. Done right, this would be very useful indeed from my side as a more abstract person who�s always looking for applications.
% I agree with Dan that it would be good to have a tutorial on diagrammatic reasoning, as well as a user�s guide to operads and props. It would be great to have also a tutorial on available software for diagrammatic reasoning, as e.g. Globular.
% On the other hand, it might be good to have also crash courses on (selected  topics in) category theory, network theory, dynamical systems, systems biology, linguistics and AI that will take place in the days preceding the actual workshop, aimed at students and young researchers, people from industry who are interested in learning more about ACT but are new either to category theory, or the applications, or both.

\subsection{Online seminar}
To supplement the tutorial weekend, we will host an online seminar called the Kan Extension Lab targeted at a mix of graduate students: 50\% should be graduate students already familiar with category theory, and 50\% should be graduate students from other fields with some specific application in mind that might benefit from category theory. The two groups will then be matched into teams of 2-3 members.

The seminar will be run over the 16 weeks immediately prior to the workshop. Biweekly, a team of participants will lecture on and develop an extant application of category theory in their field, usually based on a published paper. After each lecture, the presenters will write a blog post summary, and the other participants will be asked to comment on the post. The blog posts will be then aggregated into a survey of applied category theory and reformatted as a significant component of the expository/survey publication first discussed in Section 2.1, on ``a multi-disciplinary research program in which concepts, structures, and methods from one discipline can be reused in another''. There will be a special session of the tutorial weekend to celebrate the end of the seminar, and to discuss the proposed publication. There will also be an opportunity to present the work developing during the seminar at the workshop week. % The output of the Extension Lab will be published in a special issue of \redout{???}.

This seminar is based on a series of online seminars on pure category theory called the Kan Extension Seminar (itself based on the original Kan Seminars at MIT), organized by Emily Riehl, Alexander Campbell, and Brendan Fong (one of the organizers of this workshop). 

% The Kan Extension Seminar, piloted in early 2014, was conceived as an online (�extension�) Kan seminar for peridoctoral category theorists. A dozen category theorists plus one facilitator met biweekly for videochat presentations on classic papers in the field. After each seminar meeting, the presenter wrote a blog post summary that was published on the n-Category Caf�. Some reflections on the first iteration of the Kan Extension Seminar can be found in the December 2014 issue of the Notices of the AMS.


\section{Participants}
The estimated number of participants is 55. At the time of writing of this proposal, the planned workshop already has ?? confirmed participants with affiliations in the Netherlands, the UK, the US, France, Germany, Italy, and [...].

Confirmed participants for tutorial weekend: \redout{??}. 

Junior/senior ratio: \redout{??}.

\begin{enumerate}
\item Samson Abramsky (Oxford), prof, computer science
\item John Baez (UC Riverside), prof, math
\item Bob Coecke (Oxford), prof, computer science
\item Brendan Fong (MIT), postdoc, computer science
\item Joshua Tan (Oxford), PhD, computer science
\item \redout{... [copy over from spreadsheet when complete.]}
\end{enumerate}

\section{Factsheet}
A separate factsheet has been attached to this application.

\section{Budget}
A separate budget has been attached to this application.

\bibliographystyle{plain}
\bibliography{lorentz} 

\end{document}
